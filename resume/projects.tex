%%%%%% TITLE %%%%%%

\EN{\cvsection[3]{Projects}}
\DE{\cvsection{Projekte}}

%%%%%% COMMANDS %%%%%%

\newcommand*{\gh}[2]{
  \href{https://github.com/#1}{{\faGithub} {
    \EN{{#2} on GitHub}
    \DE{{#2} auf GitHub}
  }}
}

\newcommand*{\gplay}[2]{
  \href{https://play.google.com/store/apps/details?id=#1}{{\faGooglePlay} {
    \EN{{#2} on Google Play}
    \DE{{#2} auf Google Play}
  }}
}

\newcommand*{\ghs}[1]{
  \href{https://github.com/#1}{{\faGithub}}
}

\newcommand*{\tgs}[1]{
  \href{https://t.me/#1}{{\faTelegram}}
}

\newcommand*{\npms}[1]{
  \href{https://npm.im/#1}{{\faNpm}}
}

\newcommand{\wip}{
  { \color{red} \textbf{\textsuperscript{WIP}} }
}

%%%%%% ENTRIES %%%%%%

\newcommand{\projectSolar}{
  \cventry%
  {
    \EN{Operating System that tries to combine nixOS' unique features with ease of use}
    \DE{Betriebssystem dass, das NixOS Konfigurationssystem Nutzerfreundlich macht}
  }
  {
    \EN{solarOS}
    \DE{solarOS}
  }
  {alpha stage}
  {
    \EN{2020 -- 2021}
    \DE{2020 -- 2021}
  }
  {
    \begin{cvitems}
      \item {
        \ghs{ssd-solar/sunshine}
        \EN{Automatic updates for nixOS (sunshine)\wip}
        \DE{Automatische updates für NixOS (sunshine)\wip}
      }
      \item {
        \ghs{ssd-solar/distinst}
        \EN{Graphical installer for nixOS (distinst fork)\wip}
        \DE{Grafischer Installer für NixOS (distinst klon)\wip}
      }
      \item {
        \ghs{ssd-solar/dsync-service}
        \EN{Automatic synchronisation of user and os configuration (dsync)\wip}
        \DE{Automatische Synchronisation von Nutzer- und Betriebssystemkonfiguration (dsync)\wip}
      }
    \end{cvitems}
  }
}

\newcommand{\projectNixosPort}{
  \cventry%
  {
    \EN{Made different applications and services available on nixOS}
    \DE{Verschiedenste Anwendungen und Dienste für NixOS verfügbar gemacht}
  }
  {
    \EN{Ported different software to nixOS}
    \DE{Portiere verschiedenste Software für NixOS}
  }
  {ongoing}
  {
    \EN{2020 -- ongoing}
    \DE{2020 -- laufend}
  }
  {
    \begin{cvitems}
      \item {
        \ghs{NixOS/nixpkgs/tree/master/pkgs/desktops/cinnamon}
        \EN{Cinnamon Desktop Environment for nixOS}
        \DE{Cinnamon Desktop Umgebung für NixOS}
      }
      \item {
        \ghs{mkg20001/hubs-flake}
        \EN{Mozilla Hubs extended reality service for nixOS\wip}
        \DE{Mozilla Hubs Extenden Reality Dienst für NixOS}
      }
      \item {
        \ghs{mkg20001/safenet-nixos}
        \EN{Safenet Authentication Client for nixOS}
        \DE{Safenet Authentifizierungsklient für NixOS}
      }
      \item {
        \ghs{NixOS/nixpkgs/pulls?q=is\%3Apr+is\%3Amerged+author\%3Amkg20001+sort\%3Aupdated-desc+}
        \EN{Many other tiny utilities}
        \DE{Viele weitere kleine Dinge}
      }
    \end{cvitems}
  }
}

\newcommand{\projectNPM}{
  \cventry%
  {
    \EN{Different utilities to make development easier}
    \DE{Verschiedenste Werkzeuge um die Entwicklung zu vereinfachen}
  }
  {
    \EN{Various npm packages}
    \DE{Verschiedenste npm pakete}
  }
  {in production use}
  {
    \EN{2018 -- ongoing}
    \DE{2018 -- laufend}
  }
  {
    \begin{cvitems}
      \item {
        \ghs{mkg20001/it-pb-rpc}
        \npms{it-pb-rpc}
        \EN{Async Iterator utility library for length-prefixed protocol-buffers streams (it-pb-rpc) - 6k weekly downloads}
        \DE{Async Iterator Hilfbibiliothek für length-prefixed protocol-buffers streams (it-pb-rpc) - 6k wöchentliche downloades}
      }
      \item {
        \ghs{mkg20001/scriptrr}
        \npms{scriptrr}
        \EN{Library to help with calling bash scripts from node (scriptrr)}
        \DE{Bibliothek um das Aufrufen von bash Skripten von NodeJS aus zu vereinfachen (scriptrr)}
      }
      \item {
        \ghs{libp2p/js-libp2p-stardust}
        \npms{libp2p-stardust}
        \EN{(deprecated) Libp2p transport that relays connections and lets peers discover each other}
        \DE{(veraltet) Libp2p Transport der Verbindungen relayed und Knotenfindung ermöglicht}
      }
      \item {
        \ghs{mkg20001/mkgs-tool}
        \npms{mkgs-tool}
        \EN{Tool to help with managing my project configurations (mkgs-tool)}
        \DE{Werkzeug um mit der Verwaltung meiner Projektkonfigurationen zu helfen (mkgs-tool)}
      }
    \end{cvitems}
  }
}

\newcommand{\projectTGBot}{
  \cventry%
  {
    \EN{Adding different functionalities to Telegram}
    \DE{Verschiedenste Funktionalitäten zu Telegram hinzugefügt}
  }
  {
    \EN{Various Bots for Telegram}
    \DE{Verschiedenste Bots für Telegram}
  }
  {in production use}
  {
    \EN{2020 -- ongoing}
    \DE{2020 -- laufend}
  }
  {
    \begin{cvitems}
      \item {
        \ghs{mkg20001/tg-sticker-convert-bot}
        \tgs{sticker_convert_bot}
        \EN{Bot to convert image files into the format required by Telegram Stickers}
        \DE{Bot um Bilder in das Format für Telegram Sticker zu konvertieren}
      }
      % \item { \gh {mkg20001/tg-sticker-convert-bot} {Telegram Sticker Convert Bot} }
      \item {
        \ghs{mkg20001/tg-gif-export-bot}
        \tgs{gif_export_bot}
        \EN{Bot to export Telegram MP4 gifs as regular gifs}
        \DE{Bot um Telegram MP4 GIFs als normale GIFs zu exportieren}
      }
      % \item { \gh {mkg20001/tg-gif-export-bot} {Telegram Gif Export Bot} }
      \item {
        \ghs{mkg20001/tg-asticker2vid-bot}
        \tgs{asticker2vid_bot}
        \EN{Bot to export Telegram Animated Stickers as Video Files}
        \DE{Bot um animierte Telegram Sticker als Videodateien zu exportieren}
      }
      % \item { \gh {mkg20001/tg-gif-export-bot} {Telegram Animated Sticker to Video Bot} }
    \end{cvitems}
  }
}

\newcommand{\projectZeronetAndroid}{
  \cventry%
  {
    \EN{Made the ZeroNet P2P Platform available on mobile devices}
    \DE{ZeroNet P2P Platform für Mobilgeräte verfügbar gemacht}
  }
  {
    \EN{Android App for Zeronet}
    \DE{Android App für ZeroNet}
  }
  {not maintained}
  {
    \EN{2019 -- 2020}
    \DE{2019 -- 2020}
  }
  {
    \begin{cvitems}
      \item {
        \EN{4.000+ Users}
        \DE{4.000+ Nutzer}
      }
      \item { \gh {HelloZeroNet/ZeroNet-kivy} {ZeroNet Android} }
      \item { \gplay {net.mkg20001.zeronet} {ZeroNet Android} }
    \end{cvitems}
  }
}

%\newcommand{\projectXX}{
%  \cventry%
%  {
%    \EN{Project Desc}
%    \DE{}
%  }
%  {
%    \EN{Project Name}
%    \DE{}
%  }
%  {maint/success}
%  {
%    \EN{Dev time}
%    \DE{}
%  }
%  {
%    \begin{cvitems}
      % \item {
      %   \EN{}
      %   \DE{}
      % }
      % \item { \gh {mkg20001/thing} {Thing} }
%    \end{cvitems}

%  }
%}

%\newcommand{\projectXX}{
%  \cventry%
%  {
%    \EN{Project Desc}
%    \DE{}
%  }
%  {
%    \EN{Project Name}
%    \DE{}
%  }
%  {maint/success}
%  {
%    \EN{Dev time}
%    \DE{}
%  }
%  {
%    \begin{cvitems}
      % \item {
      %   \EN{}
      %   \DE{}
      % }
      % \item { \gh {mkg20001/thing} {Thing} }
%    \end{cvitems}

%  }
%}


% \cventry%
%     {Università degli Studi di Padova}
% 	{State exam for engineering profession}
% 	{Padua, IT}
% 	{2015}
% 	{}


%%%%%% LAYOUT %%%%%%

\begin{cventries}

  \projectSolar

  \projectNixosPort

  \projectNPM

  \projectTGBot

  \projectZeronetAndroid

\end{cventries}
